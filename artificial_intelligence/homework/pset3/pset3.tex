\documentclass[12pt]{article}

\usepackage{answers}
\usepackage{setspace}
\usepackage{graphicx}
\usepackage{enumitem}
\usepackage{multicol}
\usepackage{mathrsfs}
\usepackage[margin=1in]{geometry} 
\usepackage{amsmath,amsthm,amssymb}
\usepackage{pgfplots}
\usepackage{listings}
\pgfplotsset{compat=1.15}
\usepgfplotslibrary{fillbetween}
 
\newcommand{\N}{\mathbb{N}}
\newcommand{\Z}{\mathbb{Z}}
\newcommand{\C}{\mathbb{C}}
\newcommand{\R}{\mathbb{R}}

\DeclareMathOperator{\sech}{sech}
\DeclareMathOperator{\csch}{csch}
 
\newenvironment{theorem}[2][Theorem]{\begin{trivlist}
\item[\hskip \labelsep {\bfseries #1}\hskip \labelsep {\bfseries #2.}]}{\end{trivlist}}
\newenvironment{definition}[2][Definition]{\begin{trivlist}
\item[\hskip \labelsep {\bfseries #1}\hskip \labelsep {\bfseries #2.}]}{\end{trivlist}}
\newenvironment{proposition}[2][Proposition]{\begin{trivlist}
\item[\hskip \labelsep {\bfseries #1}\hskip \labelsep {\bfseries #2.}]}{\end{trivlist}}
\newenvironment{lemma}[2][Lemma]{\begin{trivlist}
\item[\hskip \labelsep {\bfseries #1}\hskip \labelsep {\bfseries #2.}]}{\end{trivlist}}
\newenvironment{exercise}[2][Exercise]{\begin{trivlist}
\item[\hskip \labelsep {\bfseries #1}\hskip \labelsep {\bfseries #2.}]}{\end{trivlist}}
\newenvironment{solution}[2][Solution]{\begin{trivlist}
\item[\hskip \labelsep {\bfseries #1}]}{\end{trivlist}}
\newenvironment{problem}[2][Problem]{\begin{trivlist}
\item[\hskip \labelsep {\bfseries #1}\hskip \labelsep {\bfseries #2.}]}{\end{trivlist}}
\newenvironment{question}[2][Question]{\begin{trivlist}
\item[\hskip \labelsep {\bfseries #1}\hskip \labelsep {\bfseries #2.}]}{\end{trivlist}}
\newenvironment{corollary}[2][Corollary]{\begin{trivlist}
\item[\hskip \labelsep {\bfseries #1}\hskip \labelsep {\bfseries #2.}]}{\end{trivlist}}
 
\begin{document}
 
% --------------------------------------------------------------
%                         Start here
% --------------------------------------------------------------
 
\title{Problem Set 2}%replace with the appropriate homework number
\author{Basil R. Yap\\ %replace with your name
50.021 Artificial Intelligence - Term 8} %if necessary, replace with your course title
\date{May 27, 2018}
\maketitle
%Below is an example of the problem environment

\section{Theory Component}
% Question 1
\begin{figure}[h!]
\includegraphics[width=\linewidth]{./assets/201806022206.png}
\end{figure}
\begin{enumerate}[label=\alph*)]
\item What is the total size of $C$'s output feature map?
\item What is the total size of $P$'s output feature map? 
\end{enumerate}
\begin{figure}[h!]
\includegraphics[width=\linewidth]{./assets/201806022208.png}
\end{figure}
\begin{enumerate}
\item[c)] How many FLOPs layer $C$ and $P$ cost in total to do one forward pass?
\end{enumerate}

\begin{solution}{}~
\begin{enumerate}[label=\alph*)]
\item Input size: $14\times14\times\times30=5880$\\


\item 
\item 
\end{enumerate}
\end{solution}

% Question 2
\begin{figure}[h!]
\includegraphics[width=\linewidth]{./assets/201806022242.png}
\includegraphics[width=\linewidth]{./assets/201806022243.png}
\end{figure}

\begin{solution}{}~

\end{solution}

\section{Coding Component}

 % \lstinputlisting[language=Python]{pset2.py}

\begin{thebibliography}{9}
\bibitem{proof1} Ng, A. (2000). CS229 Lecture notes. \emph{CS229 Lecture notes}, \emph{1}(1), 11.
\end{thebibliography}

\end{document}
